\item Al tirar tres dados se puede obtener una suma de 9 de seis formas distintas, a saber: 126;135;144;225;234;333 y una suma de 10 también de 6 formas diferentes, a saber: 136;145;226;235;244;334. Sin embargo, la experiencia dice que es más fácil obtener 10 que 9 ¿Por qué?\\
Nota: este problema lo resolvió Galileo en el siglo $XVII$ a requerimiento de un jugador, el príncipe de Toscana.\e\\
    Sean los eventos $S_9=$"la suma es 9" y $S_{10}=$"la suma es 10". Tenemos que\[
    S_9=\begin{array}{l}
        \{126,162,216,261,612,621\\
        135,153,315,351,513,531\\
        144,414,441\\
        225,252,522\\
        234,243,324,342,423,432\\
        333\}
    \end{array}\qquad S_{10}=\begin{array}{l}
        \{136,163,316,361,613,631\\
        145,154,415,451,514,541\\
        226,262,622\\
        235,253,325,352,523,532\\
        244,424,442\\
        334,343,433\}
    \end{array}
    \]
    De donde se ve que $\#S_9=25$ mientras que $\#S_10=27$, con lo cual es más probable que salga el 10.