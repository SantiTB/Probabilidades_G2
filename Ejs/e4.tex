\item En una habitación, 10 personas tienen insignias numeradas del 1 al 10. Se eligen 3 personas al azar.
    \begin{enumerate}
        \item ¿Cuál es la probabilidad de que el número menor de las insignias de las personas elegidas sea 5?\e\\
            Los casos favorables involucran una selección de la insignia 5 y dos selecciones de números mayores. Hay $1\cdot5\cdot4$ casos favorables, mientras que los casos totales son $10\cdot9\cdot8$. Entonces\[P(\text{menor de las insignias es 5})=\frac{5\cdot4}{10\cdot9\cdot8}=\frac{20}{720}=\frac{1}{36}\]
        \item ¿Cuál es la probabilidad de que el número mayor de las insignias sea 5?\e\\
            Para los casos favorables necesito sacar el 5, luego tengo que sacar dos insignias menores. Con lo cual hay $1\cdot4\cdot3$ casos favorables.\[P(\text{mayor de las insignias es 5})=\frac{4\cdot3}{10\cdot9\cdot8}=\frac{12}{720}=\frac{1}{60}\]
    \end{enumerate}