\item Cinco hombres y cinco mujeres son ordenados de acuerdo a su nota de examen. Asumimos que no hay dos notas iguales y que todos los posibles $10!$ reordenamientos son igualmente probables.
    \begin{enumerate}
        \item Encontrar la probabilidad de que una mujer obtenga el tercer puesto.\e\\
            En el tercer puesto puede haber tanto una mujer como un hombre, hay 5 casos favorables y 10 totales, entonces\[P(\text{mujer en el tercer puesto})=\frac{5}{10}=\frac{1}{2}\]
        \item Encontrar la probabilidad de que el puesto más alto alcanzado por una mujer sea el sexto.\e\\
            Para que esto sea posible, las mujeres deben ocupar los puestos 6,7,8,9 y 10. Por ende, tanto mujeres como hombres están restringidos a 5 posiciones.\[P(\text{mujeres a lo último})=\frac{5!5!}{10!}\]
    \end{enumerate}