\item En una fiesta entre 10 matrimonios (hombre-mujer), se eligen 4 de las mujeres y 4 de los hombres para formar un grupo. Hallar la probabilidad de que en ese grupo estén presentes ambos cónyuges de exactamente $k$ matrimonios, para $k=0,1,2,3,4$.\e\\
    Sean los eventos\[A_k=\text{"hay $k$ parejas"}\]
    Supongamos que arranco con $k=0$. Primero elijo 4 mujeres de entre las 10, $\binom{10}{4}$. Luego, tengo que seleccionar a 4 hombres de un grupo de 6, $\binom{6}{4}$, ya que hay 4 hombres que los tengo que descartar. Como el total de casos está dado por ${\binom{10}{4}}^2$, se tiene que\[P(A_0)=\frac{\binom{10}{4}\binom{6}{4}}{{\binom{10}{4}}^2}=\frac{1}{14}\]
    Para el caso de una pareja, puedo agarrar $\binom{10}{4}$ mujeres y puedo elegir entre $\binom{4}{1}$ para completar la pareja y luego $\binom{6}{3}$ para el resto de hombres\[P(A_1)=\frac{\binom{10}{4}\binom{4}{1}\binom{6}{3}}{{\binom{10}{4}}^2}=\frac{8}{21}\]
    El resto se argumenta de manera similar obteniendo\begin{align*}
        P(A_2)=\frac{\binom{10}{4}\binom{4}{2}\binom{6}{2}}{{\binom{10}{4}}^2}&=\frac{3}{7}\\
        P(A_3)=\frac{\binom{10}{4}\binom{4}{3}\binom{6}{1}}{{\binom{10}{4}}^2}&=\frac{4}{35}\\
        P(A_4)=\frac{\binom{10}{4}\binom{4}{4}\binom{6}{0}}{{\binom{10}{4}}^2}&=\frac{1}{210}
    \end{align*}